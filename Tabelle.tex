 \documentclass[ngerman]{scrartcl} %lädt die Dokumentklasse
                                %Artikel von Koma-Skript, als Option
                                %übergebe ich, dass ich die Trennung
                                %nach neuer deutscher Rechtschreibung
                                %wünsche
 \begin{document}
 \section{Überschrift 1}         %erste Gliederungsebene
 \label{sec:uberschrift-1}       %label wenn wir später auf diese
                                %Überschrift verweisen wollen
 \LaTeXe{}
 \begin{table}

 \begin{tabular}{ l|c|c|c|c}
    \hline
 $\epsilon$& a [keV]&$\Delta$a [keV] &b &$\Delta$b \\
 \hline
 $\epsilon_{mean}$&$0.04276$ &$0.0001$ &$2.742$ &$0.001$ \\
 \hline
 $\epsilon_{Fit}$&$0.07636$ &$0.0001$ &$1.807$ &$0.001$ \\
\end{tabular}
\end{table}

\begin{table}
	\begin{tabular}{ l|c|c|c |c |c}
		Q&U [mV]&$\alpha$ &$\beta  $[1/MeV] &$\gamma $[MeV] &$\epsilon $[MeV]  \\ 
		\hline
		
		$\frac{20}{90}$ &5 &$0.2467$  &$0.4875$ & $0.5691$&$0.0014$  \\
		\hline
		$\frac{20}{90}$ &10 &$0.147$ &$0.2585$ &$0.3988$ &$0.0049$  \\
		\hline
		$\frac{20}{90}$ &5 &$0.2374$ &$0.4232$ &$0.5947$ &$0.0023$  \\
		\hline
		$\frac{16}{90}$ &5 &$0.3415$ &$0.3678$ &$0.6805$ &$0.0016$  \\
		\hline
	\end{tabular}
\end{table}

\begin{table}
	
	\begin{tabular}{c|c| c|c|c|c |c |c|c|c}
		A$ [^{\circ}]$&A [rad]&$\Phi_{meas} [^{\circ}]$& $\Phi_{meas} [rad]$      &$\delta$T [keV]         & $\Delta \delta$T [keV]      &$\delta \Phi_{korr}$[rad]         & $\Delta \delta \Phi_{korr}$[rad]       & $\epsilon$[keV]        & $\delta \epsilon $ [keV]     \\ \hline
-34,12 & -0,5955 & 31,8134 & 0,5552 & 0 & 2,4932 & -0,0461 & 0,0094 & -7,5143 & 3,1359 \\ \hline
-34,12 & -0,5955 & 26,8534 & 0,4687 & 0 & 2,4932 & 0,0405 & 0,0094 & -5,803 & 2,7456 \\ \hline
-31,56 & -0,5508 & 27,2934 & 0,4764 & 12,9502 & 2,4932 & 0,0221 & 0,0094 & -2,7227 & 1,8482 \\ \hline
-31,56 & -0,5508 & 32,8734 & 0,5737 & 12,9502 & 2,4932 & -0,0753 & 0,0094 & -17,3389 & 4,9201 \\ \hline
-28,75 & -0,5018 & 33,4733 & 0,5842 & 27,579 & 2,4932 & -0,0979 & 0,0094 & -26,6208 & 6,3466 \\ \hline
-28,75 & -0,5018 & 25,9632 & 0,4531 & 27,579 & 2,4932 & 0,0331 & 0,0094 & -7,2871 & 3,0181 \\ \hline
-25,18 & -0,4395 & 25,2032 & 0,4399 & 46,7353 & 2,4932 & 0,0305 & 0,0094 & -9,5 & 3,3857 \\ \hline
-25,18 & -0,4395 & 33,0232 & 0,5764 & 46,7353 & 2,4932 & -0,106 & 0,0094 & -27,0306 & 6,5984 \\ \hline
-20,01 & -0,3492 & 28,6332 & 0,4997 & 75,4666 & 2,4932 & -0,0532 & 0,0094 & -3,2556 & 1,8528 \\ \hline
-37,03 & -0,6463 & 31,3434 & 0,547 & -14,2503 & 2,4932 & -0,0261 & 0,0094 & -3,6863 & 2,155 \\ \hline
-37,03 & -0,6463 & 26,9734 & 0,4708 & -14,2503 & 2,4932 & 0,0502 & 0,0094 & -6,9877 & 3,0396 \\ \hline
-40,24 & -0,7023 & 27,7834 & 0,4849 & -29,3466 & 2,4932 & 0,0486 & 0,0094 & -4,9106 & 2,5272 \\ \hline
-40,24 & -0,7023 & 33,2834 & 0,5809 & -29,3466 & 2,4932 & -0,0474 & 0,0094 & -12,8492 & 4,0562 \\ \hline
-43,36 & -0,7568 & 34,5534 & 0,6031 & -43,35 & 2,4932 & -0,0579 & 0,0094 & -21,0245 & 5,209 \\ \hline
-43,36 & -0,7568 & 27,7435 & 0,4842 & -43,35 & 2,4932 & 0,0609 & 0,0094 & -7,0029 & 3,0798 \\ \hline
-46,4 & -0,8098 & 28,5935 & 0,4991 & -56,3191 & 2,4932 & 0,0568 & 0,0094 & -4,8099 & 2,4849 \\ \hline
-46,4 & -0,8098 & 32,8235 & 0,5729 & -56,3191 & 2,4932 & -0,017 & 0,0094 & -6,7228 & 2,7354 \\ \hline
-50 & -0,8727 & 29,0635 & 0,5073 & -70,7652 & 2,4932 & 0,0606 & 0,0094 & -4,8126 & 2,4785 \\ \hline
-50 & -0,8727 & 31,8836 & 0,5565 & -70,7652 & 2,4932 & 0,0114 & 0,0094 & -2,5919 & 1,2493 \\ \hline
-55 & -0,9599 & 33,2737 & 0,5807 & -89,0873 & 2,4932 & 0,0023 & 0,0094 & -6,5713 & 2,3587 \\ \hline
-55 & -0,9599 & 30,11 & 0,5255 & -89,0873 & 2,4932 & 0,0576 & 0,0094 & -3,6929 & 1,9761 \\ \hline
	\end{tabular}
\end{table}




 Zeiteichung:
  f(x) = p1*x + p2
  
  $  p1 =     0,06243 \pm 0.00079$
  $ p2 =         1.8  \pm 0,434$
 $\chi^2=19,8734$
 
                    %\LaTeXe{} ist ein Befehl der das Logo
                                %für LaTeX2e ausgibt
 \end{document}